
    \begin{frame}
        \maketitle
    \end{frame}
    
    % Use starred version (e.g. \section*{Section name})
    % to disable (sub)section page.
    \section{Aufbau des Models}
    
    \begin{frame}{Graphennetzwerk}
        Graph:
        \begin{itemize}
            \item Knoten: geographisch/ politisch getrennte Populationen
            \item Kanten: Austausch zwischen Populatinen
        \end{itemize}
%        \includegraphics[scale=0.55]{sample_graph.pdf}
    \end{frame}
    

    \begin{frame}{Population}

         
        \begin{itemize}
            \item gut durchmischtes System 
            \item 5 Kompartmente: 
                \begin{itemize}
                    \item S (susceptible)
                    \item E (exposed)
                    \item I (infected)
                    \item R (recovered)
                    \item D (deceased)
                \end{itemize}  
            \item SEIR Model: Übergang zwischen Kompartmenten
        \end{itemize}

    \end{frame}

    

    \begin{frame}{SEIR Model}
        DGL Parameter: 
        \begin{itemize}
            \item Verbreitungsrate (Wahrscheinlichkeit Weitergabe)
            \item Inkubationszeit
            \item Krankheitsdauer
            \item Fatalität 
            \item demographische Parameter: Geburten- und Sterberate
            \item Dauer bis Verlust von Immunität
        \end{itemize}
    
    \end{frame}




    
    % \begin{frame}{Kanten}
    % \begin{columns}[t, onlytextwidth]
    %         \column{0.3\textwidth}
    %                     Austauschgewichte
    %                     \begin{itemize}
    %                         \item abhängig vom Knoten
    %                         \item abhängig vom Kompartment
    %                     \end{itemize}

            
    %         \column{0.5\textwidth}
            
    %             \begin{figure}
    %          \includegraphics[scale=0.35]{sample_graph.pdf}
    %             \end{figure}
    %     \end{columns}
    % \end{frame}
    
    \begin{frame}{Kanten}
        Austauschgewichte
        \begin{itemize}
            \item abhängig vom Knoten
            \item abhängig vom Kompartment
        \end{itemize}

        \begin{figure}
%            \includegraphics[scale=0.4]{sample_graph.pdf}
        \end{figure}
    \end{frame}


    \begin{frame}{Paramterraum}
        \begin{itemize}
            \item Netzwerkgröße
            \item Vernetzungsgrad
            \item Netzwekstruktur
            \item DGL Paramter
            \item Initialisierung der Austauschparamter
            \item Initialisierung der Anfangspopulationen
        \end{itemize}
    \end{frame}
    
    
    \begin{frame}{Implementierte features}
        \begin{itemize}
            \item Lockdown
            \item lokaler Lockdown
            \item Impfung
            \item konstante Infektionsquelle
            \item einzelne Infektionsquelle 
        \end{itemize}
    \end{frame}

    \section{Analyse}
    
    \begin{frame}{Erkunden des Systems}
        \begin{itemize}
            \item Vergleich der globalen Netzwerk Werte mit einem Referenzsystem
            \item Ausbreitung der Krankheit im Netzwerk
            \item \dots
        \end{itemize}
    \end{frame}
    
    
    \begin{frame}{Vergleich der globalen Netzwerk Werte mit einem Referenzsystem}
    Referenzsystem: 
    \begin{itemize}
        \item Anfangsbedingungen \newline 
        = Addition der Populationen an den Knoten
        \item DGL Paramter \newline 
        =  Mittelwerte von den Knoten Parametern 
    \end{itemize}
    \end{frame}

    
    
%     \begin{frame}[plain]{Plain frame}
%         This is a frame with plain style and it is numbered.
%     \end{frame}
    
    
%     \begin{frame}{Ausbreitung der Krankheit im Netzwerk}
        
%     \end{frame}
    
%     \begin{frame}{Blocks}
%         \begin{block}{Block}
%             Text.
%         \end{block}
%         \pause
%         \begin{alertblock}{Alert block}
%             Alert \alert{text}.
%         \end{alertblock}
%         \pause
%         \begin{exampleblock}{Example block}
%             Example \textcolor{example}{text}.
%         \end{exampleblock}
%     \end{frame}
    
%     \begin{frame}{Lists}
%         \begin{columns}[t, onlytextwidth]
%             \column{0.33\textwidth}
%                 Items:
%                 \begin{itemize}
%                     \item Item 1
%                     \begin{itemize}
%                         \item Subitem 1.1
%                         \item Subitem 1.2
%                     \end{itemize}
%                     \item Item 2
%                     \item Item 3
%                 \end{itemize}
            
%             \column{0.33\textwidth}
%                 Enumerations:
%                 \begin{enumerate}
%                     \item First
%                     \item Second
%                     \begin{enumerate}
%                         \item Sub-first
%                         \item Sub-second
%                     \end{enumerate}
%                     \item Third
%                 \end{enumerate}
            
%             \column{0.33\textwidth}
%                 Descriptions:
%                 \begin{description}
%                     \item[First] Yes.
%                     \item[Second] No.
%                 \end{description}
%         \end{columns}
%     \end{frame}


% \setbeamertemplate{caption}[numbered]
%     \begin{frame}{Figures and Tables}
%         \begin{columns}
%             \column{0.4\textwidth}
%                 \begin{figure}
%                     \centering
%                     \includegraphics{focuslogo.pdf}
%                     \caption{Figure caption.}
%                     \label{fig:focuslogo}
%                 \end{figure}
                
%             \column{0.6\textwidth}
%                 \begin{table}
%                     \centering
%                     \begin{tabular}{rcc}
%                          & Heading 1 & Heading 2 \\\hline
%                         Row 1 & \(v_{11}\) & \(v_{12}\) \\
%                         Row 2 & \(v_{21}\) & \(v_{22}\) \\
%                         Row 3 & \(v_{31}\) & \(v_{32}\) \\
%                     \end{tabular}
%                     \caption{Table caption.}
%                     \label{tab:demo}
%                 \end{table}
%         \end{columns}
%     \end{frame}
    
%     \begin{frame}[focus]
%         Thanks for using \textbf{Focus}!
%     \end{frame}
    
%     \appendix
%     \begin{frame}{References}
%         \nocite{*}
%         \bibliography{demo_bibliography}
%         \bibliographystyle{plain}
%     \end{frame}
    
%     \begin{frame}{Backup frame}
%         \usebeamercolor[fg]{normal text}
%         This is a backup frame, useful to include additional material for questions from the audience.
%         \vfill
%         The package \texttt{appendixnumberbeamer} is used not to number appendix frames.
%     \end{frame}
    
    % \begin{frame}[t]
    %     This frame has an empty title and is aligned to top.
    % \end{frame}
    

    % \begin{frame}[noframenumbering]{No frame numbering}
    %     This frame is not numbered and is citing reference \cite{knuth74}.
    % \end{frame}
    
    % \begin{frame}{Typesetting and Math}
    %     The packages \texttt{inputenc} and \texttt{FiraSans}\footnote{\url{https://fonts.google.com/specimen/Fira+Sans}}\textsuperscript{,}\footnote{\url{http://mozilla.github.io/Fira/}} are used to properly set the main fonts.
    %     \vfill
    %     This theme provides styling commands to typeset \emph{emphasized}, \alert{alerted}, \textbf{bold}, \textcolor{example}{example text}, \dots
    %     \vfill
    %     \texttt{FiraSans} also provides support for mathematical symbols:
    %     \begin{equation*}
    %         e^{i\pi} + 1 = 0.
    %     \end{equation*}
    % \end{frame}
